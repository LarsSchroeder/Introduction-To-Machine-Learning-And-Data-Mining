\subsection{Algorytm}
Na wejściu algorytmu zostanie podany graf w formacie przedstawionym w pkt. 4. 
\\Następujący pseudokod prezentuje przebieg algorytmu:
\begin{description}

	\item[Pseudokod] \hfill
	\begin{enumerate}[1.]
		\item Wczytaj graf G
		\item Utwórz algorytmem Kruskala minimalne drzewo rozpinające:
			\begin{enumerate}[A.]
				\item Utwórz las $L$ z wierzchołków grafu $G$ – każdy wierzchołek jest na początku osobnym drzewem.
				\item Utwórz zbiór $S$ zawierający wszystkie krawędzie grafu $G$.
				\item Uporządkuj zbiór $S$\ w kolejności rosnącej.
				\item Dopóki $S$ nie jest pusty:
					\begin{enumerate}[a.]
						\item Pobierz krawędź o minimalnej wadze z $S$ i przypisz do $e$.
						\item Jeśli $e$ łączy dwa różne drzewa:
						\begin{enumerate}[i.]
						 	\item dodaj $e$ do lasu $L$, tak aby połączyła dwa odpowiadające drzewa w jedno.
						\item Jeśli $L$ jest drzewem rozpinającym idź do kroku 3.
						 \end{enumerate}
					\end{enumerate}
			\end{enumerate}
		\item przejdź drzewo $L$ i utwórz z niego cykl Hamiltona
			\begin{enumerate}[A.]
				\item $root$ := wybierz korzeń drzewa $L$.
				\item $H$ = MetodaA(L, root).
				\item dodaj krawędź od ostatniego wierzchołka do korzenia grafu $H$.
			\end{enumerate}
		\item zwróć rozwiązanie $H$
	\end{enumerate}

	\item[Opis funkcji pomocniczych] \hfill \\
	Rozwiązaniem będziemy nazywać listę wierzchołków generowaną przez metody A i B, która wskazuje kolejność przechodzenia wierzchołków w drzewie.
	\begin{description}
	\item[Metoda A] \hfill \\ 
	 Funkcja przechodzi przez podrzewo zaczynając w korzeniu $w$ i kończąc na jego dziecku.\\
	 Rozwiązanie MetodaA(Wierzchołek $w$):
	\begin{enumerate}[1.]
		\item Jeśli drzewo o wierzchołku $w$ ma <= 3 wierzchołki:
				\begin{enumerate}[A.]
				\item zwróć przejście metodą A podstawowego grafu i zakończ.
				\end{enumerate}
		\item Utwórz puste rozwiązanie $r$.
		\item Dla każdego dziecka $d$ wierzchołka $w$:
			\begin{enumerate}[A.]
			\item Dodaj do $r$ rozwiązanie znalezione przez $MetodaB(d)$.
			\item Wierzchołek $n$ = pobierz następne dziecko wierzchołka $w$.
			\item Jeśli $n$ nie jest puste:
				\begin{enumerate}[a.]
				\item Do $r$ dodaj pierwsze dziecko wierzchołka $n$ jeśli istnieje lub wierzchołek $n$.
				\end{enumerate}
			\end{enumerate}
		\item Zwróć $r$.
	\end{enumerate}
	
	\item[Metoda B] \hfill \\ 
	Funkcja przechodzi przez podrzewo o korzeniu $w$ zaczynając na jego dziecka i kończąc na nim.\\
	Rozwiązanie MetodaB(Wierzchołek $w$):
	\begin{enumerate}[1.]
		\item Jeśli drzewo o wierzchołku $w$ ma <= 3 wierzchołki:
				\begin{enumerate}[A.]
				\item zwróć przejście metodą B podstawowego grafu i zakończ.
				\end{enumerate}
		\item Utwórz puste rozwiązanie $r$.
		\item Dla każdego dziecka $d$ wierzchołka $w$:
			\begin{enumerate}[a.]
			\item Dodaj do $r$ rozwiązanie znalezione przez $MetodaA(d)$.
			\item Wierzchołek $n$ = pobierz następne dziecko wierzchołka $w$
			\item Jeśli $n$ nie jest puste:
				\begin{enumerate}[i.]
				\item Do $r$ dodaj wierzchołek $n$ jeśli istnieje i idź do pkt. 4.
				\end{enumerate}
			\item Do $r$ dodaj wierzchołek $w$.
			\end{enumerate}
		\item Zwróć $r.$
	\end{enumerate}

	\end{description}

\end{description}